
\documentclass{article}
\usepackage[pdftex]{graphicx}
\usepackage{rotating}
\usepackage{amsmath}
\usepackage{authblk}
\usepackage{float}
\usepackage{multirow}
\usepackage{fullpage}
\usepackage{Sweave}
\usepackage{pdflscape}
\usepackage{amsmath}   
\usepackage{amssymb}
\usepackage{wasysym}
\usepackage{ctable}
\newcommand{\fg}[3]{\begin{figure}[htbp]%
        \leavevmode%
        \centerline{\includegraphics{#1}}%
        \caption{#3}%
        \label{#2}%
        \end{figure}}
%\fgh{basefilename}{label}{caption}{short caption for TOC}
\newcommand{\fgh}[4]{\begin{figure}[htbp]%
        \leavevmode%
        \centerline{\includegraphics{#1}}%
        \caption[#4]{#3}%
        \label{#2}%
        \end{figure}}
\begin{document} 
\Sconcordance{concordance:Individualprojectreport.tex:Individualprojectreport.Rnw:%
1 98 1 1 18 2 1 1 2 13 0 1 10 16 1 1 19 1 2 13 0 1 2 13 1 %
1 16 1 1 1 2 13 0 1 2 141 1}



\title{CPH576C Individual Project Report}
%\author[1]{\small{Bruce Barber}}
\author{\small{Meng Lu}}

\affil{\footnotesize{GIDP Statistics \\ Email: menglu@email.arizona.edu}}
%\affil[2]{\footnotesize{Department of Pediatric, The University of Arizona}}
%\affil[2]{\footnotesize{Statistical Consulting Laboratory \\ email: zlu@arizona.edu}}

\maketitle
\begin{abstract}
  
\end{abstract}  
\section{INTRODUCTION}
\label{sec:Introduction}


\subsection{STUDY OBJECTIVES}
\label{sec:Study Objectives}

                
            
\subsection{STATISTICAL METHODS}
\label{sec:Statistical Methods}
    Deleting observation
    
    
    
    Missing data
    
    
    
    summary statistics
    
    
    
    Fitting data
    
    
    
    Assessment of assumptions
    
    
    
    
    Constant variance
    
    
    
    
    Sensitivity test
    
    
    Interpretation of results
    
    
    
\section{VALIDATION OF ANALYSIS}
\label{sec:Validation of Analysis}  
      
        
\section{REFERENCES}
\label{sec:References}

     
           
      




% latex table generated in R 3.1.1 by xtable 1.7-3 package
% Thu Sep 18 15:47:56 2014
\begin{table}[ht]
\centering
\begin{tabular}{rrrrr}
  \hline
 & Estimate & Std. Error & t value & Pr($>$$|$t$|$) \\ 
  \hline
(Intercept) & -0.58 & 2.80 & -0.21 & 0.8371 \\ 
  X & 15.04 & 0.48 & 31.12 & 0.0000 \\ 
   \hline
\end{tabular}
\caption{Parameter Estimates from regression model} 
\label{reg1}
\end{table} 

\section{LIST OF TABLES AND FIGURES}
\label{sec:List of Table and Figures}

     
     
     \begin{figure}[htb]
     \begin{center}
     \includegraphics[height=3in,width=3in]{/Users/mlu/Desktop/Rplot21.pdf}
     \caption{Airfreight breakage }
     \end{center}
     \end{figure}


       
      

% latex table generated in R 3.1.1 by xtable 1.7-3 package
% Thu Sep 18 15:47:56 2014
\begin{table}[ht]
\centering
\begin{tabular}{rrrrr}
  \hline
 & Estimate & Std. Error & t value & Pr($>$$|$t$|$) \\ 
  \hline
(Intercept) & 10.20 & 0.66 & 15.38 & 0.0000 \\ 
  x & 4.00 & 0.47 & 8.53 & 0.0000 \\ 
   \hline
\end{tabular}
\caption{Parameter Estimates from regression model} 
\label{reg21}
\end{table}
         
\section{APPENDICES}
\label{sec:Appendices}
          
          
           \begin{figure}[htb]
     \begin{center}
     \includegraphics[height=3in,width=3in]{/Users/mlu/Desktop/Rplot28.pdf}
     \caption{Crime rate}
     \end{center}
     \end{figure}
             
              


% latex table generated in R 3.1.1 by xtable 1.7-3 package
% Thu Sep 18 15:47:56 2014
\begin{table}[ht]
\centering
\begin{tabular}{rrrrr}
  \hline
 & Estimate & Std. Error & t value & Pr($>$$|$t$|$) \\ 
  \hline
(Intercept) & 20517.60 & 3277.64 & 6.26 & 0.0000 \\ 
  X & -170.58 & 41.57 & -4.10 & 0.0001 \\ 
   \hline
\end{tabular}
\caption{Parameter Estimates from regression model} 
\label{reg21}
\end{table}
              

\begin{table}[h]
\caption{Data for Problem $2.42$}
\centering
\begin{tabular}{c c c c c c c c}
\hline\hline
$i:$ & $1$ & $2$ & $3$ & $\cdots$ & $13$ & $14$ & $15$ \\ [0.5ex]
\hline
$Y_{i1}$ & $13.9$ & $16.0$ & $10.3$ & $\cdots$ & $14.9$ & $12.9$ & $15.8$\\
$Y_{i2}$ & $28.6$ & $34.7$ & $21.0$ &$ \cdots$ & $35.1$ & $30.0$ & $36.2$ \\[1ex]
\hline
\end{tabular}
\label{table:nonlin}
\end{table}
 


\section{problem 2.46}
  


 
   
   
   
   
   
   
   
\section{Distance}
\label{sec: distance}
We can use the distance formula
\begin{equation}
\label{eqn: distance}
	d = \sqrt{(x_2 - x_1)^2 + (y_2 - y_1)^2}
\end{equation}
to determine the distance between any two points $(x_1, y_1)$ and $(x_2, y_2)$
in $\mathbb{R}^2$.  For our example, $(x_1, y_1) = (-1, 16)$ and $(x_2, y_2) =
(3, 1)$, so plugging these values into the distance formula~\eqref{eqn:
distance} tell us the distance between the two points is
$$
	d 
	= \sqrt{(3 - (-1))^2 + (1 - 16)^2}
	= \sqrt{4^2 + (-15)^2}
	= \sqrt{241}
	.
$$

\section{Linear Fit}
\label{sec: linear fit}
Consider a linear equation $y = m x + b$ through the two points.  We will
first determine the slope $m$ of the line in Section~\ref{sec: slope}, and we
will then determine the $y$-intercept $b$ of the line in Section~\ref{sec:
intercept}.

\subsection{Slope}
\label{sec: slope}

The slope of the line passing through the two points is given by the forumula
$$
	m 
	= \frac{\Delta y}{\Delta x} 
	= \frac{y_2 - y_1}{x_2 - x_1}
	.
$$
Plugging in our two points, we find the slope of the line between them is
\begin{equation}
\label{eqn: slope}
	m 
	= \frac{1 - 16}{3 - (-1)}
	= - \frac{15}{4}
	.
\end{equation}

\subsection{Intercept}
\label{sec: intercept}

To find the $y$-intercept of the line, we start with the point-slope form of
the line of slope $m$ through the point $(x_0, y_0)$:
$$
	y - y_0 = m (x - x_0)
	.
$$
We plug in the point $(x_0, y_0) = (-1, 16)$ and the slope we found
previously~\eqref{eqn: slope} to obtain the equation
$$
	y - 16 = - \frac{15}{4} (x + 1)
	.
$$
Solving for $y$, we find the slope-intercept form of the line:
\begin{align*}
	y 
	&= - \frac{15}{4} x - \frac{15}{4} + 16 \\
	&= - \frac{15}{4} x + \frac{49}{4}
	.
\end{align*}
Therefore, the $y$-intercept is $b = 49/4$, and the equation 
$y = - \frac{15}{4} x + \frac{49}{4}$ describes the line through the two
points.

\section{Exponential Fit}
\label{sec: exponential fit}

Let us consider the exponential function $y = A e^{k x}$.  For this function
to pass through both points, we must find constants $A$ and $k$ that satisfy
both equations $16 = A e^{-k}$ and $1 = A e^{3 k}$.  To solve these two
simultaneous equations, we first take the ratio of the two equations, which
gives us a single equation involving only $k$:
$$
	16
	= \frac{A e^{-k}}{A e^{3 k}}
	= e^{-4 k}
	.
$$
We can take the natural logarithm of this equation to solve for $k$:
$$
	-4k = \ln(16) = 4 \ln (2)
	,
$$
which means $k = - \ln(2)$.

We can then use this value of $k$, along with either of the two points to
solve for $A$.  Let us consider the point $(-1, 16)$:
$$
	16 = A e^{(-\ln(2))(-1)} = A e^{\ln{2}} = 2 A
	.
$$
Solving for $A$, we find $A = 8$, and the exponential equation through both
points is
$$
	y
	= 8 e^{-\ln(2) x}
	= 8 2^{-x}
	= 8 \left( \frac{1}{2} \right)^x
	.
$$

 \end{document}

 
